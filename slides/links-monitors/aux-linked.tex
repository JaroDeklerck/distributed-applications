\section{Linked processes}

\frame{\tableofcontents[currentsection]}

\begin{frame}
    \frametitle{2 processes (\texttt{\&spawn\_link/1})}
    \begin{center}
        \begin{tikzpicture}[roundnode/.style={circle, draw=green!60, fill=green!5, very thick, minimum size=7mm}]
            %Nodes
            \node[roundnode]   (processA)                {A};
            \node[roundnode]   (processB)  [right=of processA]    {B};
            \draw[->] (processA.east) -- (processB.west);
        \end{tikzpicture}
    \end{center}
    Process A initiates link 
    \vfill
    (As process A:) \texttt{ProcessB |> Process.whereis |> Process.link}
\end{frame}

\begin{frame}
    \frametitle{2 processes (\&spawn\_link/1)}
    \begin{center}
        \begin{tikzpicture}[roundnode/.style={circle, draw=green!60, fill=green!5, very thick, minimum size=7mm}]
            %Nodes
            \node[roundnode]   (processA)                {A};
            \node[roundnode]   (processB)  [right=of processA]    {B};
            \draw[<->] (processA.east) -- (processB.west);
        \end{tikzpicture}
    \end{center}
    Processes are linked \textbf{bidirectionally}
\end{frame}

\begin{frame}
    \frametitle{2 processes (\&spawn\_link/1)}
    \begin{center}
        \begin{tikzpicture}[roundnode/.style={circle, draw=green!60, fill=green!5, very thick, minimum size=7mm}]
            \node[roundnode]   (processA)                {A};
            \node[cross out, draw]   (processB)  [right=of processA]    {B};
            \draw[<->] (processA.east) -- (processB.west);
        \end{tikzpicture}
    \end{center}

    Process B dies
\end{frame}

\begin{frame}
    \frametitle{2 processes (\&spawn\_link/1)}
    \begin{center}
        \begin{tikzpicture}[roundnode/.style={circle, draw=green!60, fill=green!5, very thick, minimum size=7mm}]
            \node[cross out, draw]   (processA)                {A};
            \node[cross out, draw]   (processB)  [right=of processA]    {B};
            \draw[<->] (processA.east) -- (processB.west);
        \end{tikzpicture}
    \end{center}

    Process A receives a "death message", and if it is not a system process, no custom behaviour can be implemented when this message is received.
    \vfill
    \textit{Note: Even if it is a system process, the end result should also be death. 
    This is often used when you want to do certain actions, such as cleaning up your process.}
\end{frame}
