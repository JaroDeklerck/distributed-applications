\section{Internals}

\frame{\tableofcontents[currentsection]}

\begin{frame}
    \frametitle{Elixir Processes}
    \structure{Other Languages}
    \begin{itemize}
        \item Use threads sparingly
        \item Thread creation is costly affair
        \item Mitigated by using thread pools
        \item Synchronization required
        \item No language support
    \end{itemize}
    \vskip5mm
    \structure{Elixir}
    \begin{itemize}
        \item Create as many processes as you want
        \item No synchronization problems
        \item Processes are cheap
    \end{itemize}
\end{frame}

\begin{frame}
    \frametitle{Internals}
    \begin{center}
        \begin{tikzpicture}[scale=0.9,transform shape,
                            os process/.style={drop shadow,fill=red!50},
                            os process header/.style={opacity=0.5,font=\huge\sc},
                            os thread/.style={drop shadow,fill=blue!50},
                            os thread header/.style={opacity=0.5,font=\large\sc},
                            elixir process/.style={drop shadow,fill=green!50}]
            \draw[os process] (0,0) rectangle (10, 8);
            \node[os process header,anchor=north] at (5,8) {os process};

            \foreach \x/\y in {0.5/0.5,5.5/0.5,0.5/4,5.5/4} {
                \coordinate (p) at (\x,\y);
                \draw[os thread] (p) rectangle ++(4, 3);
                \node[os thread header,anchor=north] at ($ (p) + (2,3) $) {os thread};

                \foreach[evaluate={\i*0.5-0.25} as \dx] \i in {1,...,8} {
                    \foreach[evaluate={\j*0.5-0.2} as \dy] \j in {1,...,5} {
                        \draw[elixir process] (\x+\dx,\y+\dy) circle[radius=0.1cm];
                    }
                }
            }

            \only<2>{
                \node[note] at (5,4) {
                    Each dot = Elixir process
                };
            }
        \end{tikzpicture}
    \end{center}
\end{frame}

\begin{frame}
    \frametitle{Internals}
    \begin{itemize}
        \item Virtual Machine lives in single OS process
        \item It creates $N$ threads on a machine with $N$ cores
        \item Processes are distributed among these threads
        \item Processes run "in bits"
        \item Cooperative "multithreading" makes it efficient
    \end{itemize}
\end{frame}
