\documentclass{ucll-slides}
\usepackage{pxfonts}
\usepackage[utf8]{inputenc}
\usepackage{tikz}
\usepackage{calc}
\usepackage{ucll-code}


\usetikzlibrary{calc,shadows,tikzmark}

\coursename{Distributed Applications}
\title{intro}


\begin{document}

\maketitle

\section{Course overview}

\begin{frame}
    \frametitle{Who we are}
    \begin{center} \large
        Wannes Fransen \\[4mm]
        Frédéric Vogels
    \end{center}
\end{frame}

\begin{frame}
    \frametitle{Classes - contacturen}
    \begin{itemize}
        \item 12 $\times$ 2 hours = 24hrs 
        \item 3stp $\Rightarrow$ 3 $\times$ 25--30hrs $\Rightarrow$ 75--90 hours 
    \end{itemize}
\end{frame}

\begin{frame}
    \frametitle{Evaluation}
    \structure{Project}
    \begin{itemize}
        \item Part 1 : 4/20 (PE with feedback)
        \item Part 2 : 6/20 (PE with feedback)
        \item Part 3 : 8/20 
    \end{itemize}
    \vskip4mm
    \structure{Oral exam}
    \begin{itemize}
        \item You can score -8, up to +2 points on the exam.
    \end{itemize}
\end{frame}

\begin{frame}
    \frametitle{Distributed system - According to wikipedia}

    While there is no single definition of a distributed system, the following defining properties are commonly used as:
    \begin{itemize}
        \item There are several autonomous computational entities (computers or nodes), each of which has its own local memory.
        \item The entities communicate with each other by message passing.
    \end{itemize}
    
\end{frame}

\begin{frame}
    \frametitle{Language}
    Elixir 

    \begin{itemize}
        \item Erlang VM
        \item Functional
        \item Process-oriented
        \item Fault-tolerant
    \end{itemize}
\end{frame}

\end{document}